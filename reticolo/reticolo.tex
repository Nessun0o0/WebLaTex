\documentclass{article}
\usepackage{graphicx} % Required for inserting images

% Language setting
% Replace `english' with e.g. `spanish' to change the document language
\usepackage[english]{babel}
\usepackage{caption}
\usepackage{subcaption}
\usepackage{float}
\usepackage{siunitx}
\usepackage{multicol}
\usepackage{gensymb}
\usepackage[T1]{fontenc}
\usepackage[utf8]{inputenc}
\usepackage{blindtext}
\usepackage[export]{adjustbox}

% Bobliography package

\usepackage{biblatex}
\addbibresource{sample.bib}

% Set page size and margins
% Replace `letterpaper' with `a4paper' for UK/EU standard size
\usepackage[a4paper,top=2cm,bottom=2cm,left=3cm,right=3cm,marginparwidth=1.75cm]{geometry}

% Useful packages
\usepackage{amsmath}
\usepackage[colorlinks=true, allcolors=blue]{hyperref}
\usepackage{wrapfig}

\title{Spettrometro a reticolo}
\author{Francesco Palombo, Davide Russillo, Emanuele Rizzo}
\date{14 October 2024}

\begin{document}

\maketitle

\begin{abstract}
    L'esperienza si propone la misura di alcune righe dello spettro di emissione di una sorgente al mercurio tramite un reticolo di diffrazione tarato con il doppietto del sodio. Il risultato ottenuto è $\lambda_{colore} = ( a \pm b ) \si{\angstrom}$.
\end{abstract}

\section{Operazioni preliminari}
Le misure degli angoli di deviazione delle righe sono effettuate tramite uno spettrometro. E' importante che l'oculare dello spettrometro lavori con raggi che arrivano dall'infinito, perciò è regolato per mettere a fuoco oggetti ad una quindicina di metri (dettagli dell'edificio di fronte al laboratorio), distanza abbastanza grande per considerare il regime di far field. Una volta orientato il cannocchiale nella direzione del collimatore e portata l'immagine della fenditura al centro dell'oculare, si regola la distanza tra la fenditura del collimatore e la lente convergente in modo da ottenere un'immagine nitida.

\section{Ortogonalizzazione del reticolo}
Descrizione procedimento?
Errori sistematici. ciao.

\section{Passo del reticolo con doppietto Na}
Descrizione?
Errori sistematici

\section{Spettro Hg e misura delle varie lunghezze d'onda}
"Ordine delle misure"

\section{Elaborazione dati}

\section{Analisi degli errori}

\section{Bonus: caratterizzazione del reticolo del Caccianiga}

\section{Conclusioni}

\newpage
\section{Appendice}
\subsection{Foto}
\subsection{Dati Grezzi}
\subsection{Grafici}

\end{document}